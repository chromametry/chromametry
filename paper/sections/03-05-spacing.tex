\subsection{Spacing Uniformity ($\mathcal{U}$)}

Spacing between adjacent color steps is measured using the
\textbf{CIEDE2000 color-difference metric}~\cite{luo2001}.
For a sequential color scale of $N$ ordered steps
$\{c_0, c_1, \dots, c_{N-1}\}$, differences are computed only between
consecutive entries:
\begin{equation}
\delta_i = \Delta E_{00}(c_{i-1}, c_i),
\qquad i = 1, \dots, N-1.
\end{equation}

An ideally spaced color scale exhibits approximately constant perceptual
increments, i.e., $\delta_i \approx \delta_j$ for all $i,j$.
To quantify relative dispersion of these increments independently of their
absolute magnitude, spacing uniformity is evaluated using the
\textbf{coefficient of variation (CV)}:
\begin{equation}
CV = \frac{\sigma(\{\delta_i\})}{\mu(\{\delta_i\})},
\end{equation}
where $\mu(\cdot)$ and $\sigma(\cdot)$ denote the mean and standard deviation
of the set $\{\delta_i\}$, respectively. As a dimensionless quantity,
$CV$ provides a scale-invariant measure of non-uniformity.

Since $CV$ is unbounded above and inversely related to quality, it is mapped
to a bounded score with intuitive directionality using the following
monotonic transform:
\begin{equation}
\mathcal{U} = \frac{1}{1 + CV}.
\end{equation}

This definition ensures $\mathcal{U} \in (0,1]$, with
$\mathcal{U} = 1$ corresponding to perfectly uniform spacing ($CV = 0$),
and $\mathcal{U}$ decreasing monotonically as dispersion increases.

A palette with uniform sampling along the color ramp therefore achieves
$\mathcal{U}$ values close to unity.
