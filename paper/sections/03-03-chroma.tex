\subsection{Chroma Smoothness ($\mathcal{S}_C$)}
\label{subsec:chroma-smoothness}

\subsubsection{Theoretical Background: Zeileis et al. (2009) Power Function Model~\cite{zeileis2009}}

Zeileis et al.~\cite{zeileis2009} propose a power-function-based parameterization to control the rate of chroma
and lightness variation along a sequential color scale, improving contrast distribution.
Using a \emph{normalized continuous position} $t \in [0,1]$, the model is defined as
\begin{equation}
\begin{cases}
H(t) = H_2 - t\,(H_1 - H_2), \\
C(t) = C_{\max} - t^{p_1}\,(C_{\max} - C_{\min}), \\
L(t) = L_{\max} - t^{p_2}\,(L_{\max} - L_{\min}),
\end{cases}
\end{equation}
where $H_1, H_2$ are hue values at the two ends of the scale;
$C_{\max}, C_{\min}$ and $L_{\max}, L_{\min}$ are chroma and lightness bounds; and
$p_1, p_2 > 0$ control the curvature of chroma and lightness variation.

\subsubsection{Derivative Discontinuity and Kink Artifacts}

Although the single power-function model guarantees monotonicity, applying it to sequential
palettes with \emph{maximum chroma near the center} typically requires piecewise definitions
(dark-to-peak and peak-to-light). The chroma derivative is
\begin{equation}
\frac{dC}{dt}
=
- p_1\, t^{p_1-1}\,(C_{\max} - C_{\min}).
\end{equation}

When parameters differ across the two segments, the first derivative becomes discontinuous at
the peak position $t_{\mathrm{peak}}$:
\begin{equation}
\lim_{t \to t_{\mathrm{peak}}^-} \frac{dC}{dt}
\neq
\lim_{t \to t_{\mathrm{peak}}^+} \frac{dC}{dt}.
\end{equation}

This lack of $C^1$ continuity produces a sharp chroma cusp, which can induce visible artifacts
(Mach bands) due to the human visual system’s sensitivity to first-derivative luminance changes~\cite{mach1865}.
This motivates the use of \emph{monotonic cubic splines}~\cite{fritsch1980}, which allow enforcing a zero
derivative at the chroma peak while preserving global monotonicity.

\subsubsection{Chroma in CIELAB Space}

Chroma in CIELAB space is computed as
\begin{equation}
C^* = \sqrt{a^{*2} + b^{*2}} .
\end{equation}
While $C^*$ is not perceptually uniform, it provides a device-independent chroma magnitude
suitable for relative comparison when properly normalized.

\subsubsection{Reference Chroma Standard ($C^*_{\mathrm{ref}}$)}

To stabilize chroma magnitude across palettes, chroma is first normalized by the maximum
chroma within the palette and subsequently rescaled into a fixed reference frame derived
from the sRGB gamut extremum in CIELAB space.

Let $C^*_i$ denote the CIELAB chroma of step $i$, and define
\begin{equation}
C^*_{\max} = \max_i C^*_i .
\end{equation}
The rescaled chroma used for smoothness evaluation is
\begin{equation}
\tilde C_i = \frac{C^*_i}{C^*_{\max}} \cdot C^*_{\mathrm{ref}} .
\end{equation}

Evaluating primary and secondary vertices of the sRGB cube shows that the maximum chroma
occurs at the blue primary, yielding
\begin{equation}
C^*_{\mathrm{ref}} \approx 133.8 .
\end{equation}

\subsubsection{Ideal Chroma Trajectory}

An ideal chroma trajectory is assumed to increase monotonically from the palette start,
reach a single maximum, and then decrease monotonically toward the end.
For a discrete palette of $N$ colors indexed by $i = 0,\dots,N-1$, the ideal trajectory
$C_{\mathrm{ideal}}(i)$ is constructed using a \emph{monotonic cubic spline}~\cite{fritsch1980}
passing through three anchors:
\begin{itemize}
\item $(0, C_0)$ — start chroma,
\item $(i_{\mathrm{peak}}, \tilde C_{\max})$ — maximum chroma,
\item $(N-1, C_{N-1})$ — end chroma.
\end{itemize}
This avoids polynomial oscillation (Runge’s phenomenon)~\cite{epperson1987} and enforces smooth,
monotonic chroma variation.

\subsubsection{Smoothness Metric Definition}

Deviation from the ideal trajectory is measured using root mean square error:
\begin{equation}
RMSE =
\sqrt{
\frac{1}{N}
\sum_{i=0}^{N-1}
\left(\tilde C_i - C_{\mathrm{ideal}}(i)\right)^2
}.
\end{equation}

The maximum deviation envelope is defined as
\begin{equation}
\epsilon_{\max,i}
=
\max\!\left(
C_{\mathrm{ideal}}(i) - \tilde C_{\min},
\tilde C_{\max} - C_{\mathrm{ideal}}(i)
\right),
\end{equation}
where
$\tilde C_{\min} = \min_i \tilde C_i$ and
$\tilde C_{\max} = \max_i \tilde C_i$.

The maximum achievable error is
\begin{equation}
RMSE_{\max} =
\sqrt{
\frac{1}{N}
\sum_{i=0}^{N-1}
\epsilon_{\max,i}^2
}.
\end{equation}

The \emph{Chroma Smoothness} metric is defined as
\begin{equation}
\mathcal{S}_C
=
\max\!\left(
0,\;
1 - \frac{RMSE}{RMSE_{\max}}
\right).
\end{equation}

\subsubsection{Interpretation}

$\mathcal{S}_C \approx 1$ indicates smooth chroma progression closely matching the ideal
trajectory, while $\mathcal{S}_C \to 0$ reflects strong chroma oscillation or derivative
discontinuities. For achromatic or near-achromatic palettes where $C^*_{\max} \approx 0$,
chroma smoothness is defined as maximal, reflecting the absence of chroma variation.
