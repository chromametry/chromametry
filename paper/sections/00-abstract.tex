\begin{abstract}
The design of accessible color systems for digital interfaces currently lacks unified quantitative standards, leading to subjective and manual workflows. This document introduces a quantitative framework for evaluating sequential monochromatic color palettes by defining five principal evaluation dimensions: Contrast Efficiency, lightness linearity, chroma smoothness, hue stability, and spacing uniformity. The framework adopts the CIELAB color space for its interpretability and its role as the reference color space for computing perceptual differences using CIEDE2000. By benchmarking eleven industry-leading design systems, the framework objectively differentiates palette quality, with composite scores ranging from 59.01 to 88.69, identifying IBM Carbon and Adobe Spectrum as high-performing benchmarks. This work establishes a quantitative foundation for systematic palette evaluation and comparative analysis.
\end{abstract}

\noindent\textbf{Keywords:}
color ramp; colormap; monochromatic; color accessibility; color palette; design systems

\vspace{0.5em}
\noindent
An open-source reference implementation and benchmarks are available at\\
\url{https://github.com/chromametry/chromametry}
