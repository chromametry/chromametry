\section{Experimental Evaluation}

This section reports quantitative evaluation results for \textbf{11 widely used
design systems}, serving as a benchmark for the proposed metric framework
when applied to real-world monochromatic color ramps.

An interactive online report containing all benchmark data and visualizations
is available at:
\begin{quote}
\url{https://chromametry.github.io/chromametry/benchmarks/monochromatic}
\end{quote}

\subsection{Quantitative Analysis Results}

Table~\ref{tab:benchmark} summarizes the measured quality metrics across
all evaluated design systems. Clear differentiation is observed across both
individual metric dimensions and composite scores, indicating substantial
variation in how existing design systems balance accessibility, uniformity, and
structural consistency.

\begin{table*}[t]
\centering
\caption{Benchmark results of popular design systems evaluated using the proposed metric framework.}
\label{tab:benchmark}
\small
\begin{tabular}{lccccccccc}
\hline
Design System & Ramps & Steps & $K$ & $\eta$ & $\mathcal{L}$ & $\mathcal{S}_C$ & $\mathcal{H}$ & $\mathcal{U}$ & SCORE \\
\hline
Adobe Spectrum & 10 & 18 & 9  & 0.947 & 0.9333 & 0.8786 & 0.9138 & 0.7722 & 88.67 \\
IBM Carbon & 12 & 12 & 6  & 0.923 & 0.9303 & 0.8688 & 0.9252 & 0.7919 & 88.62 \\
U.S. Web Design System & 25 & 12 & 6 & 0.923 & 0.9359 & 0.8096 & 0.9380 & 0.7997 & 87.90 \\
Salesforce Lightning 2 & 13 & 14 & 7 & 0.933 & 0.9187 & 0.8464 & 0.9372 & 0.7107 & 86.47 \\
GitHub Primer Brand & 13 & 12 & 6 & 0.923 & 0.9243 & 0.8405 & 0.9408 & 0.6841 & 85.67 \\
Atlassian & 9 & 14 & 8 & 0.800 & 0.8964 & 0.9094 & 0.9465 & 0.7129 & 84.86 \\
Tailwind CSS & 18 & 13 & 8 & 0.789 & 0.8705 & 0.8565 & 0.9147 & 0.6780 & 81.74 \\
Ant Design & 12 & 12 & 9 & 0.711 & 0.8586 & 0.8734 & 0.9276 & 0.6550 & 79.81 \\
Material UI & 19 & 12 & 11 & 0.565 & 0.7967 & 0.7861 & 0.9239 & 0.5500 & 70.95 \\
Radix UI & 16 & 13 & 10 & 0.543 & 0.7979 & 0.7679 & 0.9481 & 0.5207 & 69.67 \\
Shopify Polaris & 12 & 17 & 15 & 0.356 & 0.7281 & 0.6892 & 0.9223 & 0.4667 & 59.86 \\
\hline
\end{tabular}
\end{table*}


Several trends emerge from these results. Design systems that tightly control
contrast span and lightness progression (e.g., IBM Carbon and Adobe Spectrum)
achieve consistently high scores across most dimensions. In contrast, systems
with large span values or uneven spacing exhibit reduced contrast efficiency
and spacing uniformity, which significantly impacts the composite score due to
the use of geometric aggregation.

\begin{quote}
\textbf{Note.} Design systems such as Bootstrap, Google Material~3, Apple Human
Interface Guidelines, and Fluent UI are excluded from this evaluation, as they
primarily define discrete semantic color tokens rather than algorithmically
constructed sequential color ramps.
\end{quote}

\subsection{Example: A Typical Report}

\begin{figure}[t]
\centering
\includegraphics[width=\linewidth]{figures/adobe-spectrum-color-palette.png}
\caption{Adobe Spectrum Color Palette. Cell value = Background Index $\pm 9$ steps (WCAG 4.5:1).}
\end{figure}

\begin{figure}[t]
\centering
\includegraphics[width=\linewidth]{figures/adobe-spectrum-palette-metrics.png}
\caption{Adobe Spectrum Palette Metrics.}
\end{figure}

\begin{figure}[t]
\centering
\includegraphics[width=\linewidth]{figures/adobe-spectrum-chart-lightness.png}
\caption{Adobe Spectrum Lightness Chart.}
\end{figure}

\begin{figure}[t]
\centering
\includegraphics[width=\linewidth]{figures/adobe-spectrum-chart-chroma.png}
\caption{Adobe Spectrum Chroma Chart.}
\end{figure}

\begin{figure}[t]
\centering
\includegraphics[width=\linewidth]{figures/adobe-spectrum-chart-hue.png}
\caption{Adobe Spectrum Hue Chart.}
\end{figure}

\begin{figure}[t]
\centering
\includegraphics[width=\linewidth]{figures/adobe-spectrum-chart-cumDeltaE00.png}
\caption{Adobe Spectrum Cumilatiive DeltaE2000 Chart.}
\end{figure}

The contrast span value $K$ is taken directly from measured \emph{Span}.
Observed density $D$ is determined by
\begin{equation}
D = \frac{K}{N - 1},
\end{equation}
where $N$ is the number of palette steps.

\begin{table}[t]
\centering
\caption{Experimental values of $K$, $D$, and Contrast Efficiency $\eta$.}
\label{tab:efficiency-values}
\small
\begin{tabular}{lcccc}
\hline
Color Palette & $N$ & $K$ & $D$ & $\eta$ \\
\hline
Adobe Spectrum & 18 & 9 & 0.500 & 0.947 \\
IBM Carbon & 12 & 6 & 0.500 & 0.923 \\
U.S. Web Design System & 12 & 6 & 0.500 & 0.923 \\
Salesforce Lightning 2 & 14 & 7 & 0.500 & 0.933 \\
GitHub Primer Brand & 12 & 6 & 0.500 & 0.923 \\
Atlassian & 14 & 8 & 0.571 & 0.800 \\
Tailwind CSS & 13 & 8 & 0.615 & 0.789 \\
Ant Design & 12 & 9 & 0.750 & 0.711 \\
Material UI & 12 & 11 & 0.917 & 0.565 \\
Radix UI & 13 & 10 & 0.769 & 0.543 \\
Shopify Polaris & 17 & 15 & 0.882 & 0.356 \\
\hline
\end{tabular}
\end{table}


This table shows that design systems achieving high efficiency all have low
density, while palettes with excessively wide span lead to wasted lightness
space and are strongly penalized by metric $\eta$.

Thus, contrast span $K$ is observed consistently across popular industry
palettes, though previously not systematically identified or exploited. With
these results, users can select color pairs for background and text with
distance $K$ without requiring runtime verification.

\subsection{Discussion}

\subsubsection{High Score Group (Score $>$ 85)}

IBM Carbon, Adobe Spectrum, and USWDS maintain consistently high indicators
across all aspects.
\begin{itemize}
\item \textbf{Observation:} High Contrast Efficiency ($\eta > 0.9$) and
Lightness Linearity ($\mathcal{L} > 0.93$).
\item \textbf{Analysis:} Span distance $K$ maintains optimal ratio to total
steps $N$, ensuring high density of WCAG-compliant color pairs.
\end{itemize}

\subsubsection{Medium Score Group (75--85)}

Tailwind CSS and Ant Design show disparities among component indicators.
\begin{itemize}
\item \textbf{Observation:} High Chroma Smoothness
($\mathcal{S}_C > 0.84$) but lower Contrast Efficiency.
\item \textbf{Analysis:} Large span relative to $N$ reduces simultaneous
usability of color pairs.
\end{itemize}

\subsubsection{Low Score Group ($<$ 75)}

Material UI (v4) and Shopify Polaris record the lowest scores.
\begin{itemize}
\item \textbf{Observation:} Low Spacing Uniformity ($\mathcal{U} < 0.6$) and
Lightness Linearity.
\item \textbf{Analysis:} Non-uniform step spacing and non-linear lightness
progression.
\end{itemize}

\subsection{Experimental Conclusions}

Benchmark data confirms the framework’s ability to classify palettes based on
physical and mathematical characteristics. Results suggest that using an even
number of steps with contrast span $\approx (N-1)/2$ yields favorable usability
properties, where every color can serve as background and a corresponding text
color always exists satisfying WCAG~4.5:1.
