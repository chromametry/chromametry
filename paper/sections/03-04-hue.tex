\subsection{Hue Stability ($\mathcal{H}$)}

The \textbf{Hue Stability} metric evaluates the consistency of hue in a monochromatic
color scale as lightness varies. A palette is considered stable if hue fluctuates
minimally around a reference color rather than drifting toward other hues.

\subsubsection{Hue Representation and Periodicity Handling}

Hue in CIELAB space is determined by
\begin{equation}
h_i = \mathrm{atan2}(b^*_i, a^*_i).
\end{equation}

Since hue is a periodic angular quantity, the hue sequence is \textbf{unwrapped}
to remove discontinuities at $360^\circ$ and ensure continuity:
\begin{equation}
h^{\mathrm{unwrap}}_i =
h^{\mathrm{unwrap}}_{i-1} + \Delta h_i,
\qquad
\Delta h_i \in (-180^\circ, 180^\circ].
\end{equation}

This unwrapping ensures that hue deviations reflect actual shifts rather than
artifacts from angular periodicity.

\subsubsection{Reference Color and Hue Deviation}

The reference hue $h_{\mathrm{base}}$ is chosen as the hue of the
\textbf{color with maximum chroma} in the palette, as this typically represents
the dominant hue identity.

The angular distance between each step and the reference color is defined as
\begin{equation}
d_i =
\min\!\left(
|h_i - h_{\mathrm{base}}|,
\; 360^\circ - |h_i - h_{\mathrm{base}}|
\right),
\end{equation}
where $d_i \in [0, 180^\circ]$ represents the hue deviation at step $i$.

\subsubsection{Normalization by Worst-Case Linear Drift}

For normalization, a conservative worst-case scenario is defined in which hue
drifts linearly from the reference hue toward its complementary hue. Although hue
values are unwrapped to ensure numerical continuity, hue distance remains bounded
by $180^\circ$, which represents the maximum distinguishable angular deviation in
circular hue space.

The maximum deviation at step $i$ in this scenario is
\begin{equation}
d^{\max}_i = 180^\circ \cdot \frac{i}{N - 1}.
\end{equation}

This value serves as the \textbf{theoretical upper bound} for hue drift at each
position.

\subsubsection{Hue Stability Metric Definition}

The root mean square error of hue deviation is computed as
\begin{equation}
RMSE =
\sqrt{
\frac{1}{N}
\sum_{i=0}^{N-1} d_i^2
}.
\end{equation}

The corresponding worst-case error is
\begin{equation}
RMSE_{\max} =
\sqrt{
\frac{1}{N}
\sum_{i=0}^{N-1} \left(d^{\max}_i\right)^2
}.
\end{equation}

The \textbf{Hue Stability} metric is defined by
\begin{equation}
\mathcal{H} =
\max\!\left(
0,\;
1 - \frac{RMSE}{RMSE_{\max}}
\right).
\end{equation}

\subsubsection{Interpretation}

\begin{itemize}
  \item $\mathcal{H} \approx 1$: Hue remains stable around the reference color
  throughout the palette.
  \item $\mathcal{H} \to 0$: Strong hue drift approaching the worst-case linear
  scenario.
\end{itemize}

This metric is independent of lightness and chroma, measuring only the
\textbf{geometric stability of hue} in color space.
