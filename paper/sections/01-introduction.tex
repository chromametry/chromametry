\section{Introduction}

Designing color systems for digital interfaces requires balancing uniformity with accessibility constraints. In practice, the absence of a unified quantitative evaluation framework leads to several key limitations:

\begin{itemize}
  \item \textbf{High subjectivity:}
  Color palette quality assessment remains largely perceptual, lacking objective measurement indicators for comparing different systems.

  \item \textbf{Automation barriers:}
  Without standardized metrics, the generation of systematic color scales ($M$ hues $\times$ $N$ steps) remains predominantly manual. Existing interpolation methods struggle to preserve linearity and accessibility in the absence of quantitative validation criteria.

  \item \textbf{Combinatorial contrast testing:}
  Verifying WCAG-compliant contrast across all color pairs is operationally inefficient, often requiring exhaustive inspection during both design and deployment stages.

  \item \textbf{Poor maintainability and scalability:}
  Color systems without quantitative foundations face difficulties when extending palette ranges or performing synchronized updates while preserving brand-specific characteristics.
\end{itemize}

This work aims to shift color system design from a heuristic-driven process toward a metrics-based methodology, establishing a quantitative reference for systematic palette evaluation and comparison.
