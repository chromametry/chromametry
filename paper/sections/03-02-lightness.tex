\subsection{Lightness Linearity ($\mathcal{L}$)}

The \textbf{Lightness Linearity} metric evaluates the degree to which
\emph{perceptual} lightness progression in a palette follows a linear
trend with step index. Unlike pure $L^*$, lightness is affected by the
Helmholtz--Kohlrausch effect~\cite{high2023}, where high-chroma colors
are perceived as lighter than achromatic colors with the same $L^*$.

\subsubsection{Equivalent Achromatic Lightness}

To compensate for this effect, we use \textbf{Equivalent Achromatic
Lightness (EAL)} according to High et al.~\cite{high2023}:

\[
L_{\mathrm{EAL}} =
L^* + \bigl(f_{BY}(h) + f_R(h)\bigr)\, C^*
\]

where chroma and hue in CIELAB space are determined by

\[
C = \sqrt{a^{*2} + b^{*2}}
\]

\[
h = \operatorname{atan2}(b^*, a^*)
\]

The hue-dependent correction functions are given by

\[
f_{BY}(h) =
0.1644 \left|
\sin\!\left(\frac{h - 90^\circ}{2}\right)
\right| + 0.0603
\]

\[
f_R(h) =
\begin{cases}
0.1307 \, |\cos(h)| + 0.0060,
& h \in [0^\circ, 90^\circ] \cup [270^\circ, 360^\circ],\\
0,
& \text{otherwise}.
\end{cases}
\]

\subsubsection{Linear Regression by Step Index}

For a hue family of $N$ colors ordered by \textbf{monotonic lightness
progression} (either strictly increasing or strictly decreasing), let
$L_{\mathrm{EAL},i}$ denote the EAL value at step $i$.
A linear model is fitted using \textbf{ordinary least squares}:

\[
\hat{L}_{\mathrm{EAL}}(i) = \alpha i + \beta,
\qquad i = 0, \dots, N-1.
\]

Here, the sign of $\alpha$ implicitly captures the direction of
progression: $\alpha > 0$ corresponds to increasing lightness, while
$\alpha < 0$ corresponds to decreasing lightness.

The root mean square error is computed as

\[
RMSE =
\sqrt{
\frac{1}{N}
\sum_{i=0}^{N-1}
\bigl(
L_{\mathrm{EAL},i} -
\hat{L}_{\mathrm{EAL}}(i)
\bigr)^2
}.
\]

\subsubsection{Normalization by Fitted Range}

To normalize error independently of absolute lightness magnitude and
progression direction, error is normalized by the
\textbf{extremal values of the fitted line}:

\[
L_{\min}^{\text{fit}} =
\min\!\bigl(
\hat{L}_{\mathrm{EAL}}(0),\;
\hat{L}_{\mathrm{EAL}}(N-1)
\bigr).
\]

\[
L_{\max}^{\text{fit}} =
\max\!\bigl(
\hat{L}_{\mathrm{EAL}}(0),\;
\hat{L}_{\mathrm{EAL}}(N-1)
\bigr).
\]


At each step $i$, the maximum allowable deviation within this fitted
range is

\[
\epsilon_i^{\max} =
\max\!\left(
\left|
\hat{L}_{\mathrm{EAL}}(i) - L_{\min}^{\text{fit}}
\right|,
\left|
L_{\max}^{\text{fit}} - \hat{L}_{\mathrm{EAL}}(i)
\right|
\right).
\]

From this, the maximum normalized error is defined as

\[
RMSE_{\max} =
\sqrt{
\frac{1}{N}
\sum_{i=0}^{N-1}
\bigl(
\epsilon_i^{\max}
\bigr)^2
}.
\]

This normalization constrains the metric to the envelope of the fitted
linear model and is invariant to whether the palette progresses from
dark-to-light or light-to-dark, ensuring score stability independent of
absolute position or direction on the $L^*$ axis.

\subsubsection{Metric Definition}

The \textbf{Lightness Linearity} metric is defined by

\[
\mathcal{L} =
\max\!\left(
0,\;
1 - \frac{RMSE}{RMSE_{\max}}
\right).
\]

The value $\mathcal{L} \in [0,1]$ approaches 1 when lightness progression
closely follows a linear trend; strong local fluctuations or
non-linearity will decrease the score.

In degenerate cases where the fitted line magnitude is negligible
(palette nearly flat in lightness), the metric is conventionally set to
$\mathcal{L} = 1$.

\subsubsection{Interpretation}

This metric evaluates \textbf{consistency of lightness progression}
rather than purely geometric $L^*$. Using EAL allows $\mathcal{L}$ to
more accurately reflect users' visual perception, especially for
high-chroma palettes where the Helmholtz--Kohlrausch effect plays a
significant role.
