\subsection{Contrast Efficiency ($\eta$)}
\label{subsec:contrast-efficiency}

Contrast Efficiency evaluates the economy of index separation required to satisfy accessibility constraints. To establish an ideal upper bound for accessibility, this framework references a \textbf{neutral gray ramp} ($a^* = 0, b^* = 0$).

The choice of an achromatic scale is motivated by its inherent stability: neutral colors provide the most direct mapping between relative luminance $Y$ and perceived lightness $L^*$, as they are unaffected by the Helmholtz--Kohlrausch effect—where high chroma increases perceived brightness despite constant luminance~\cite{high2023}. Furthermore, neutral scales occupy the achromatic axis of the color gamut, minimizing non-linearities induced by gamut mapping constraints common in highly saturated hues.

In this ideal achromatic case, $L^*$ is related to relative luminance $Y \in [0,1]$ by the CIE 1976 transform~\cite{iso11664}:

\[
L^*(Y) =
\begin{cases}
116\,Y^{1/3} - 16, & Y > \left(\frac{6}{29}\right)^3 \\
116 \left(\frac{29}{6}\right)^2 Y + 16, & Y \le \left(\frac{6}{29}\right)^3
\end{cases}
\]

Applying the WCAG 4.5:1 boundary conditions:
\begin{itemize}
\item For $Y_{\min}=0$ (black background), the required $Y_{\text{text}} = 0.175 \Rightarrow L^* \approx 48.9$.
\item For $Y_{\max}=1$ (white background), the required $Y_{\text{text}} \approx 0.183 \Rightarrow L^* \approx 49.9$.
\end{itemize}

Accordingly, the conservative lightness threshold is defined as
\[
L^*_{\text{target}} \approx 49.9.
\]

Since the CIELAB lightness axis spans $[0,100]$, this corresponds to a normalized lightness ratio
\[
\lambda = \frac{L^*_{\text{target}}}{100} \approx 0.50.
\]

Although the exact luminance-derived ratio yields $\lambda \approx 0.49$, this work adopts a rounded value $\lambda = 0.50$ to avoid false numerical precision and to maintain symmetry along the CIELAB lightness axis.

For a palette with $N$ discrete steps, the \textbf{ideal density} is defined as:
\[
D_{\text{ideal}} = \lambda \cdot \frac{N-1}{N}
\]

For practical design guidance, the corresponding \textbf{ideal contrast span} is:
\[
K_{\text{ideal}} = \lceil \lambda \cdot (N-1) \rceil
\]

Note that $D_{\text{ideal}}$ is used for metric calculation to ensure smooth behavior, while $K_{\text{ideal}}$ serves as a discrete design target. For example, with $N=12$:
\begin{itemize}
\item $D_{\text{ideal}} = 0.5 \times \frac{11}{12} \approx 0.458$
\item $K_{\text{ideal}} = \lceil 5.5 \rceil = 6$
\end{itemize}

This formulation ensures smooth metric behavior across different palette sizes and eliminates the aliasing effects inherent in ceiling-based calculations. As $N \to \infty$, $D_{\text{ideal}} \to \lambda = 0.5$.

Let $D$ be the actual contrast density, where $K$ is the measured contrast span. Contrast Efficiency $\eta$ is defined by:
\[
\eta =
\begin{cases}
1, & D \le D_{\text{ideal}} \\
0, & D \ge 1.0 \\
1 - \dfrac{D - D_{\text{ideal}}}{1.0 - D_{\text{ideal}}}, & \text{otherwise}
\end{cases}
\]

\begin{table}[t]
\centering
\caption{Reference values: $K_{\text{ideal}}$ (design target) and $D_{\text{ideal}}$ (metric calculation)}
\label{tab:contrast-efficiency-ideal}
\begin{tabular}{c c c c}
\hline
Steps ($N$) & Formula & $K_{\text{ideal}}$ & $D_{\text{ideal}}$ \\
\hline
10 & $\lceil 0.50 \times 9 \rceil$  & 5 & 0.450 \\
11 & $\lceil 0.50 \times 10 \rceil$ & 5 & 0.455 \\
12 & $\lceil 0.50 \times 11 \rceil$ & 6 & 0.458 \\
13 & $\lceil 0.50 \times 12 \rceil$ & 6 & 0.462 \\
14 & $\lceil 0.50 \times 13 \rceil$ & 7 & 0.464 \\
15 & $\lceil 0.50 \times 14 \rceil$ & 7 & 0.467 \\
16 & $\lceil 0.50 \times 15 \rceil$ & 8 & 0.469 \\
17 & $\lceil 0.50 \times 16 \rceil$ & 8 & 0.471 \\
18 & $\lceil 0.50 \times 17 \rceil$ & 9 & 0.472 \\
\hline
\end{tabular}
\end{table}

