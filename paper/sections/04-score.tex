\section{Composite Quality Score}

To aggregate multiple quality metrics into a single scalar score without
allowing strong dimensions to compensate for weak ones, the
\textbf{geometric mean} is used instead of the arithmetic mean.
The composite score is defined as
\begin{equation}
\mathrm{SCORE}
=
100 \cdot
\left(
\prod_{k=1}^{5} (M_k + \varepsilon)
\right)^{1/5},
\end{equation}
where
\[
M_k \in \{\eta, \mathcal{L}, \mathcal{S}_C, \mathcal{H}, \mathcal{U}\}
\]
denote the five normalized component metrics, each bounded in $[0,1]$, and
$\varepsilon = 10^{-6}$ is a small constant introduced solely to ensure
numerical stability when any component metric approaches zero.

The use of the geometric mean has several desirable properties:
\begin{itemize}
\item \textbf{Strong penalty for imbalance}: a single poor metric significantly
reduces the overall score, preventing compensation by unrelated dimensions.
\item \textbf{Encouragement of uniform quality} across accessibility,
perceptual uniformity, and structural consistency criteria.
\item \textbf{Scale invariance} with respect to the component metrics,
preserving their relative contributions.
\end{itemize}

Since all component metrics are bounded in $[0,1]$, the resulting SCORE is
guaranteed to lie in the interval $[0,100]$, enabling direct and intuitive
comparison across different palette generation methods.

Accordingly, SCORE reflects overall palette completeness rather than excellence
in any single isolated attribute.
