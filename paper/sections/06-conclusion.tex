\section{Conclusion}

This work introduces a unified quantitative framework for evaluating the
quality of sequential monochromatic color palettes, grounded in the CIELAB
color space and motivated by both perceptual uniformity and accessibility
requirements. Five complementary metrics are proposed, covering contrast
efficiency under WCAG~4.5:1 constraints, lightness linearity, chroma
smoothness, hue stability, and perceptual spacing uniformity. Each metric
captures a distinct structural or perceptual property and is normalized to
enable consistent aggregation.

Through large-scale benchmarking of widely used industry design systems, the
framework reveals systematic differences in how palettes allocate lightness
range, distribute perceptual steps, and preserve hue identity. In particular,
the analysis identifies contrast span $K$ as a previously under-formalized
but critical parameter, with empirical results indicating that spans close to
$(N-1)/2$ maximize accessibility efficiency while preserving usable color
density. These findings demonstrate that many high-quality palettes converge
toward similar structural ratios, despite differing design origins.

The proposed composite score, based on geometric aggregation, enables
objective comparison without allowing compensation between weak and strong
dimensions, thereby reflecting overall palette completeness rather than
isolated excellence. As a result, the framework supports reproducible
benchmarking, automated palette validation, and optimization-driven color
system design.

This study focuses on monochromatic sequential color ramps and adopts the
CIELAB color space as a unified evaluation domain, ensuring direct
compatibility with the CIEDE2000 color-difference metric used for perceptual
uniformity assessment. Future work will extend the framework to diverging and
multivariate palettes, and investigate how analogous metric formulations may
be adapted to alternative perceptually uniform color spaces (e.g., CAM02-UCS
or CAM16) while preserving cross-space comparability.
